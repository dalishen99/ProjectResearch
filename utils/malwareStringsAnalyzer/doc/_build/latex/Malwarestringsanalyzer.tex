% Generated by Sphinx.
\def\sphinxdocclass{report}
\newif\ifsphinxKeepOldNames \sphinxKeepOldNamestrue
\documentclass[letterpaper,10pt,oneside]{sphinxmanual}
\usepackage{iftex}

\ifPDFTeX
  \usepackage[utf8]{inputenc}
\fi
\ifdefined\DeclareUnicodeCharacter
  \DeclareUnicodeCharacter{00A0}{\nobreakspace}
\fi
\usepackage{cmap}
\usepackage[T1]{fontenc}
\usepackage{amsmath,amssymb,amstext}
\usepackage[english]{babel}
\usepackage{times}
\usepackage[Bjarne]{fncychap}
\usepackage{longtable}
\usepackage{sphinx}
\usepackage{multirow}
\usepackage{eqparbox}


\addto\captionsenglish{\renewcommand{\figurename}{Fig.\@ }}
\addto\captionsenglish{\renewcommand{\tablename}{Table }}
\SetupFloatingEnvironment{literal-block}{name=Listing }

\addto\extrasenglish{\def\pageautorefname{page}}




\title{Malware strings analyzer Documentation}
\date{Aug 22, 2016}
\release{0.1}
\author{Yann Ferrere}
\newcommand{\sphinxlogo}{}
\renewcommand{\releasename}{Release}
\makeindex

\makeatletter
\def\PYG@reset{\let\PYG@it=\relax \let\PYG@bf=\relax%
    \let\PYG@ul=\relax \let\PYG@tc=\relax%
    \let\PYG@bc=\relax \let\PYG@ff=\relax}
\def\PYG@tok#1{\csname PYG@tok@#1\endcsname}
\def\PYG@toks#1+{\ifx\relax#1\empty\else%
    \PYG@tok{#1}\expandafter\PYG@toks\fi}
\def\PYG@do#1{\PYG@bc{\PYG@tc{\PYG@ul{%
    \PYG@it{\PYG@bf{\PYG@ff{#1}}}}}}}
\def\PYG#1#2{\PYG@reset\PYG@toks#1+\relax+\PYG@do{#2}}

\expandafter\def\csname PYG@tok@gd\endcsname{\def\PYG@tc##1{\textcolor[rgb]{0.63,0.00,0.00}{##1}}}
\expandafter\def\csname PYG@tok@gu\endcsname{\let\PYG@bf=\textbf\def\PYG@tc##1{\textcolor[rgb]{0.50,0.00,0.50}{##1}}}
\expandafter\def\csname PYG@tok@gt\endcsname{\def\PYG@tc##1{\textcolor[rgb]{0.00,0.27,0.87}{##1}}}
\expandafter\def\csname PYG@tok@gs\endcsname{\let\PYG@bf=\textbf}
\expandafter\def\csname PYG@tok@gr\endcsname{\def\PYG@tc##1{\textcolor[rgb]{1.00,0.00,0.00}{##1}}}
\expandafter\def\csname PYG@tok@cm\endcsname{\let\PYG@it=\textit\def\PYG@tc##1{\textcolor[rgb]{0.25,0.50,0.56}{##1}}}
\expandafter\def\csname PYG@tok@vg\endcsname{\def\PYG@tc##1{\textcolor[rgb]{0.73,0.38,0.84}{##1}}}
\expandafter\def\csname PYG@tok@vi\endcsname{\def\PYG@tc##1{\textcolor[rgb]{0.73,0.38,0.84}{##1}}}
\expandafter\def\csname PYG@tok@mh\endcsname{\def\PYG@tc##1{\textcolor[rgb]{0.13,0.50,0.31}{##1}}}
\expandafter\def\csname PYG@tok@cs\endcsname{\def\PYG@tc##1{\textcolor[rgb]{0.25,0.50,0.56}{##1}}\def\PYG@bc##1{\setlength{\fboxsep}{0pt}\colorbox[rgb]{1.00,0.94,0.94}{\strut ##1}}}
\expandafter\def\csname PYG@tok@ge\endcsname{\let\PYG@it=\textit}
\expandafter\def\csname PYG@tok@vc\endcsname{\def\PYG@tc##1{\textcolor[rgb]{0.73,0.38,0.84}{##1}}}
\expandafter\def\csname PYG@tok@il\endcsname{\def\PYG@tc##1{\textcolor[rgb]{0.13,0.50,0.31}{##1}}}
\expandafter\def\csname PYG@tok@go\endcsname{\def\PYG@tc##1{\textcolor[rgb]{0.20,0.20,0.20}{##1}}}
\expandafter\def\csname PYG@tok@cp\endcsname{\def\PYG@tc##1{\textcolor[rgb]{0.00,0.44,0.13}{##1}}}
\expandafter\def\csname PYG@tok@gi\endcsname{\def\PYG@tc##1{\textcolor[rgb]{0.00,0.63,0.00}{##1}}}
\expandafter\def\csname PYG@tok@gh\endcsname{\let\PYG@bf=\textbf\def\PYG@tc##1{\textcolor[rgb]{0.00,0.00,0.50}{##1}}}
\expandafter\def\csname PYG@tok@ni\endcsname{\let\PYG@bf=\textbf\def\PYG@tc##1{\textcolor[rgb]{0.84,0.33,0.22}{##1}}}
\expandafter\def\csname PYG@tok@nl\endcsname{\let\PYG@bf=\textbf\def\PYG@tc##1{\textcolor[rgb]{0.00,0.13,0.44}{##1}}}
\expandafter\def\csname PYG@tok@nn\endcsname{\let\PYG@bf=\textbf\def\PYG@tc##1{\textcolor[rgb]{0.05,0.52,0.71}{##1}}}
\expandafter\def\csname PYG@tok@no\endcsname{\def\PYG@tc##1{\textcolor[rgb]{0.38,0.68,0.84}{##1}}}
\expandafter\def\csname PYG@tok@na\endcsname{\def\PYG@tc##1{\textcolor[rgb]{0.25,0.44,0.63}{##1}}}
\expandafter\def\csname PYG@tok@nb\endcsname{\def\PYG@tc##1{\textcolor[rgb]{0.00,0.44,0.13}{##1}}}
\expandafter\def\csname PYG@tok@nc\endcsname{\let\PYG@bf=\textbf\def\PYG@tc##1{\textcolor[rgb]{0.05,0.52,0.71}{##1}}}
\expandafter\def\csname PYG@tok@nd\endcsname{\let\PYG@bf=\textbf\def\PYG@tc##1{\textcolor[rgb]{0.33,0.33,0.33}{##1}}}
\expandafter\def\csname PYG@tok@ne\endcsname{\def\PYG@tc##1{\textcolor[rgb]{0.00,0.44,0.13}{##1}}}
\expandafter\def\csname PYG@tok@nf\endcsname{\def\PYG@tc##1{\textcolor[rgb]{0.02,0.16,0.49}{##1}}}
\expandafter\def\csname PYG@tok@si\endcsname{\let\PYG@it=\textit\def\PYG@tc##1{\textcolor[rgb]{0.44,0.63,0.82}{##1}}}
\expandafter\def\csname PYG@tok@s2\endcsname{\def\PYG@tc##1{\textcolor[rgb]{0.25,0.44,0.63}{##1}}}
\expandafter\def\csname PYG@tok@nt\endcsname{\let\PYG@bf=\textbf\def\PYG@tc##1{\textcolor[rgb]{0.02,0.16,0.45}{##1}}}
\expandafter\def\csname PYG@tok@nv\endcsname{\def\PYG@tc##1{\textcolor[rgb]{0.73,0.38,0.84}{##1}}}
\expandafter\def\csname PYG@tok@s1\endcsname{\def\PYG@tc##1{\textcolor[rgb]{0.25,0.44,0.63}{##1}}}
\expandafter\def\csname PYG@tok@ch\endcsname{\let\PYG@it=\textit\def\PYG@tc##1{\textcolor[rgb]{0.25,0.50,0.56}{##1}}}
\expandafter\def\csname PYG@tok@m\endcsname{\def\PYG@tc##1{\textcolor[rgb]{0.13,0.50,0.31}{##1}}}
\expandafter\def\csname PYG@tok@gp\endcsname{\let\PYG@bf=\textbf\def\PYG@tc##1{\textcolor[rgb]{0.78,0.36,0.04}{##1}}}
\expandafter\def\csname PYG@tok@sh\endcsname{\def\PYG@tc##1{\textcolor[rgb]{0.25,0.44,0.63}{##1}}}
\expandafter\def\csname PYG@tok@ow\endcsname{\let\PYG@bf=\textbf\def\PYG@tc##1{\textcolor[rgb]{0.00,0.44,0.13}{##1}}}
\expandafter\def\csname PYG@tok@sx\endcsname{\def\PYG@tc##1{\textcolor[rgb]{0.78,0.36,0.04}{##1}}}
\expandafter\def\csname PYG@tok@bp\endcsname{\def\PYG@tc##1{\textcolor[rgb]{0.00,0.44,0.13}{##1}}}
\expandafter\def\csname PYG@tok@c1\endcsname{\let\PYG@it=\textit\def\PYG@tc##1{\textcolor[rgb]{0.25,0.50,0.56}{##1}}}
\expandafter\def\csname PYG@tok@o\endcsname{\def\PYG@tc##1{\textcolor[rgb]{0.40,0.40,0.40}{##1}}}
\expandafter\def\csname PYG@tok@kc\endcsname{\let\PYG@bf=\textbf\def\PYG@tc##1{\textcolor[rgb]{0.00,0.44,0.13}{##1}}}
\expandafter\def\csname PYG@tok@c\endcsname{\let\PYG@it=\textit\def\PYG@tc##1{\textcolor[rgb]{0.25,0.50,0.56}{##1}}}
\expandafter\def\csname PYG@tok@mf\endcsname{\def\PYG@tc##1{\textcolor[rgb]{0.13,0.50,0.31}{##1}}}
\expandafter\def\csname PYG@tok@err\endcsname{\def\PYG@bc##1{\setlength{\fboxsep}{0pt}\fcolorbox[rgb]{1.00,0.00,0.00}{1,1,1}{\strut ##1}}}
\expandafter\def\csname PYG@tok@mb\endcsname{\def\PYG@tc##1{\textcolor[rgb]{0.13,0.50,0.31}{##1}}}
\expandafter\def\csname PYG@tok@ss\endcsname{\def\PYG@tc##1{\textcolor[rgb]{0.32,0.47,0.09}{##1}}}
\expandafter\def\csname PYG@tok@sr\endcsname{\def\PYG@tc##1{\textcolor[rgb]{0.14,0.33,0.53}{##1}}}
\expandafter\def\csname PYG@tok@mo\endcsname{\def\PYG@tc##1{\textcolor[rgb]{0.13,0.50,0.31}{##1}}}
\expandafter\def\csname PYG@tok@kd\endcsname{\let\PYG@bf=\textbf\def\PYG@tc##1{\textcolor[rgb]{0.00,0.44,0.13}{##1}}}
\expandafter\def\csname PYG@tok@mi\endcsname{\def\PYG@tc##1{\textcolor[rgb]{0.13,0.50,0.31}{##1}}}
\expandafter\def\csname PYG@tok@kn\endcsname{\let\PYG@bf=\textbf\def\PYG@tc##1{\textcolor[rgb]{0.00,0.44,0.13}{##1}}}
\expandafter\def\csname PYG@tok@cpf\endcsname{\let\PYG@it=\textit\def\PYG@tc##1{\textcolor[rgb]{0.25,0.50,0.56}{##1}}}
\expandafter\def\csname PYG@tok@kr\endcsname{\let\PYG@bf=\textbf\def\PYG@tc##1{\textcolor[rgb]{0.00,0.44,0.13}{##1}}}
\expandafter\def\csname PYG@tok@s\endcsname{\def\PYG@tc##1{\textcolor[rgb]{0.25,0.44,0.63}{##1}}}
\expandafter\def\csname PYG@tok@kp\endcsname{\def\PYG@tc##1{\textcolor[rgb]{0.00,0.44,0.13}{##1}}}
\expandafter\def\csname PYG@tok@w\endcsname{\def\PYG@tc##1{\textcolor[rgb]{0.73,0.73,0.73}{##1}}}
\expandafter\def\csname PYG@tok@kt\endcsname{\def\PYG@tc##1{\textcolor[rgb]{0.56,0.13,0.00}{##1}}}
\expandafter\def\csname PYG@tok@sc\endcsname{\def\PYG@tc##1{\textcolor[rgb]{0.25,0.44,0.63}{##1}}}
\expandafter\def\csname PYG@tok@sb\endcsname{\def\PYG@tc##1{\textcolor[rgb]{0.25,0.44,0.63}{##1}}}
\expandafter\def\csname PYG@tok@k\endcsname{\let\PYG@bf=\textbf\def\PYG@tc##1{\textcolor[rgb]{0.00,0.44,0.13}{##1}}}
\expandafter\def\csname PYG@tok@se\endcsname{\let\PYG@bf=\textbf\def\PYG@tc##1{\textcolor[rgb]{0.25,0.44,0.63}{##1}}}
\expandafter\def\csname PYG@tok@sd\endcsname{\let\PYG@it=\textit\def\PYG@tc##1{\textcolor[rgb]{0.25,0.44,0.63}{##1}}}

\def\PYGZbs{\char`\\}
\def\PYGZus{\char`\_}
\def\PYGZob{\char`\{}
\def\PYGZcb{\char`\}}
\def\PYGZca{\char`\^}
\def\PYGZam{\char`\&}
\def\PYGZlt{\char`\<}
\def\PYGZgt{\char`\>}
\def\PYGZsh{\char`\#}
\def\PYGZpc{\char`\%}
\def\PYGZdl{\char`\$}
\def\PYGZhy{\char`\-}
\def\PYGZsq{\char`\'}
\def\PYGZdq{\char`\"}
\def\PYGZti{\char`\~}
% for compatibility with earlier versions
\def\PYGZat{@}
\def\PYGZlb{[}
\def\PYGZrb{]}
\makeatother

\renewcommand\PYGZsq{\textquotesingle}

\begin{document}

\maketitle
\tableofcontents
\phantomsection\label{index::doc}



\chapter{/}
\label{index:welcome-to-malware-strings-analyzer-s-documentation}\label{index:id1}

\section{msa.py}
\label{index:module-msa}\label{index:msa-py}\index{msa (module)}\index{exit\_error() (in module msa)}

\begin{fulllineitems}
\phantomsection\label{index:msa.exit_error}\pysiglinewithargsret{\sphinxcode{msa.}\sphinxbfcode{exit\_error}}{}{}
Call the usage() function and exit msa.py with an error return
value.

\end{fulllineitems}

\index{getopts() (in module msa)}

\begin{fulllineitems}
\phantomsection\label{index:msa.getopts}\pysiglinewithargsret{\sphinxcode{msa.}\sphinxbfcode{getopts}}{\emph{argv}}{}
Parse and extract set options.
\begin{quote}\begin{description}
\item[{Parameters}] \leavevmode
\textbf{\texttt{argv}} (\emph{\texttt{List}}) -- List of parameters set by the user.

\item[{Returns}] \leavevmode
All valid parameters entered by the user or exit in case of

\end{description}\end{quote}

unknown parameter.
:rtype: List

\end{fulllineitems}

\index{start() (in module msa)}

\begin{fulllineitems}
\phantomsection\label{index:msa.start}\pysiglinewithargsret{\sphinxcode{msa.}\sphinxbfcode{start}}{\emph{argv}}{}
Analyze users options and run the malware string analyzer.
\begin{quote}\begin{description}
\item[{Parameters}] \leavevmode
\textbf{\texttt{argv}} -- Users options.

\item[{Type}] \leavevmode
List

\item[{Returns}] \leavevmode
True if msa success, False otherwise.

\item[{Return type}] \leavevmode
Boolean

\end{description}\end{quote}

\end{fulllineitems}

\index{usage() (in module msa)}

\begin{fulllineitems}
\phantomsection\label{index:msa.usage}\pysiglinewithargsret{\sphinxcode{msa.}\sphinxbfcode{usage}}{}{}
Display the usage message of msa.py

\end{fulllineitems}



\chapter{/lib/}
\label{index:lib}

\section{vt.py}
\label{index:module-lib.vt}\label{index:vt-py}\index{lib.vt (module)}\index{Vt (class in lib.vt)}

\begin{fulllineitems}
\phantomsection\label{index:lib.vt.Vt}\pysiglinewithargsret{\sphinxstrong{class }\sphinxcode{lib.vt.}\sphinxbfcode{Vt}}{\emph{api\_key}, \emph{malware\_path}}{}
The Vt class is used in order to perform all request to the
Virus Total API.
\index{\_Vt\_\_malware\_md5() (lib.vt.Vt method)}

\begin{fulllineitems}
\phantomsection\label{index:lib.vt.Vt._Vt__malware_md5}\pysiglinewithargsret{\sphinxbfcode{\_Vt\_\_malware\_md5}}{}{}
Private method that perform the md5 algorithm on the malware binary
accessible by using the absolute path on the malware\_path attribut.
\begin{quote}\begin{description}
\item[{Returns}] \leavevmode
MD5 hash of the analyzed malware.

\item[{Return type}] \leavevmode
string

\end{description}\end{quote}

\end{fulllineitems}

\index{\_Vt\_\_print\_ip\_rslt() (lib.vt.Vt method)}

\begin{fulllineitems}
\phantomsection\label{index:lib.vt.Vt._Vt__print_ip_rslt}\pysiglinewithargsret{\sphinxbfcode{\_Vt\_\_print\_ip\_rslt}}{\emph{json}, \emph{key}, \emph{msg}}{}
Private method that display a green ``yes'' or a red ``no'' depending of
the results sent by Virus Total.
\begin{quote}\begin{description}
\item[{Parameters}] \leavevmode\begin{itemize}
\item {} 
\textbf{\texttt{json}} -- Returned json from Virus Total API.

\item {} 
\textbf{\texttt{key}} -- Json field to analyze.

\item {} 
\textbf{\texttt{msg}} -- Description of the Json field currently analyzed.

\end{itemize}

\end{description}\end{quote}

\end{fulllineitems}

\index{\_Vt\_\_send\_req() (lib.vt.Vt method)}

\begin{fulllineitems}
\phantomsection\label{index:lib.vt.Vt._Vt__send_req}\pysiglinewithargsret{\sphinxbfcode{\_Vt\_\_send\_req}}{\emph{url}, \emph{parameters}}{}
Private method that perform a HTTP request to a specific url
with parameters.
\begin{quote}\begin{description}
\item[{Parameters}] \leavevmode\begin{itemize}
\item {} 
\textbf{\texttt{url}} (\emph{\texttt{string}}) -- Contains targeted url.

\item {} 
\textbf{\texttt{parameters}} (\emph{\texttt{dict}}) -- Contains a list of parameters to be sent.

\end{itemize}

\item[{Returns}] \leavevmode
Response of the HTTP request formated in json.

\item[{Return type}] \leavevmode
dict

\end{description}\end{quote}

\end{fulllineitems}

\index{\_\_init\_\_() (lib.vt.Vt method)}

\begin{fulllineitems}
\phantomsection\label{index:lib.vt.Vt.__init__}\pysiglinewithargsret{\sphinxbfcode{\_\_init\_\_}}{\emph{api\_key}, \emph{malware\_path}}{}
Initialize api\_key and malware\_path attributs of the class.
\begin{quote}\begin{description}
\item[{Parameters}] \leavevmode\begin{itemize}
\item {} 
\textbf{\texttt{api\_key}} (\emph{\texttt{string}}) -- The user's Virus Total API key.

\item {} 
\textbf{\texttt{malware\_path}} (\emph{\texttt{string}}) -- The absolute path of the malware binary to
analyze.

\end{itemize}

\end{description}\end{quote}

\end{fulllineitems}

\index{file\_analysis() (lib.vt.Vt method)}

\begin{fulllineitems}
\phantomsection\label{index:lib.vt.Vt.file_analysis}\pysiglinewithargsret{\sphinxbfcode{file\_analysis}}{}{}
Perform a malware analysis of the selected malaware binary by using
Virus Total API.
\begin{quote}\begin{description}
\item[{Returns}] \leavevmode
Notify the success or the failure of this analysis.

\item[{Return type}] \leavevmode
Boolean

\end{description}\end{quote}

\end{fulllineitems}

\index{ip\_analysis() (lib.vt.Vt method)}

\begin{fulllineitems}
\phantomsection\label{index:lib.vt.Vt.ip_analysis}\pysiglinewithargsret{\sphinxbfcode{ip\_analysis}}{\emph{ip}}{}
Perform an ip analysis of a specified ip by using
Virus Total API.
\begin{quote}\begin{description}
\item[{Parameters}] \leavevmode
\textbf{\texttt{ip}} (\emph{\texttt{string}}) -- IP to analyze.

\item[{Returns}] \leavevmode
Notify the success or the failure of this analysis.

\item[{Return type}] \leavevmode
Boolean

\end{description}\end{quote}

\end{fulllineitems}

\index{isActivate() (lib.vt.Vt method)}

\begin{fulllineitems}
\phantomsection\label{index:lib.vt.Vt.isActivate}\pysiglinewithargsret{\sphinxbfcode{isActivate}}{}{}
Check if the api\_key has been entered by the user and consequently
if the Virus Total analysis has to be executed.
\begin{quote}\begin{description}
\item[{Returns}] \leavevmode
If the Virus Total option is activated or not.

\item[{Return type}] \leavevmode
Boolean

\end{description}\end{quote}

\end{fulllineitems}

\index{url\_analysis() (lib.vt.Vt method)}

\begin{fulllineitems}
\phantomsection\label{index:lib.vt.Vt.url_analysis}\pysiglinewithargsret{\sphinxbfcode{url\_analysis}}{\emph{url\_to\_analyze}}{}
Perform an url analysis of a specified url by using
Virus Total API.
\begin{quote}\begin{description}
\item[{Parameters}] \leavevmode
\textbf{\texttt{url\_to\_analyze}} (\emph{\texttt{string}}) -- Url to analyze.

\item[{Returns}] \leavevmode
Notify the success or the failure of this analysis.

\item[{Return type}] \leavevmode
Boolean

\end{description}\end{quote}

\end{fulllineitems}


\end{fulllineitems}



\section{rules.py}
\label{index:rules-py}\label{index:module-lib.rules}\index{lib.rules (module)}\index{Rules (class in lib.rules)}

\begin{fulllineitems}
\phantomsection\label{index:lib.rules.Rules}\pysigline{\sphinxstrong{class }\sphinxcode{lib.rules.}\sphinxbfcode{Rules}}
This object is inherited by all specific string analysis rules.
It also contained an instance of the db object, used by all rules.
\index{\_\_init\_\_() (lib.rules.Rules method)}

\begin{fulllineitems}
\phantomsection\label{index:lib.rules.Rules.__init__}\pysiglinewithargsret{\sphinxbfcode{\_\_init\_\_}}{}{}
Initialize the bar attributs that allow all rules to use
the ProgressBar object.

\end{fulllineitems}

\index{run\_analysis() (lib.rules.Rules method)}

\begin{fulllineitems}
\phantomsection\label{index:lib.rules.Rules.run_analysis}\pysiglinewithargsret{\sphinxbfcode{run\_analysis}}{\emph{string\_list}}{}
This method is used by the core to run analyze and extract strings
matching with a specific type.
\begin{quote}\begin{description}
\item[{Parameters}] \leavevmode
\textbf{\texttt{string\_list}} (\emph{\texttt{List}}) -- All strings to analyse.

\item[{Returns}] \leavevmode
A list of string without strings previously matched.

\item[{Return type}] \leavevmode
List

\end{description}\end{quote}

\end{fulllineitems}


\end{fulllineitems}



\section{progress\_bar.py}
\label{index:progress-bar-py}\label{index:module-lib.progress_bar}\index{lib.progress\_bar (module)}\index{ProgressBar (class in lib.progress\_bar)}

\begin{fulllineitems}
\phantomsection\label{index:lib.progress_bar.ProgressBar}\pysigline{\sphinxstrong{class }\sphinxcode{lib.progress\_bar.}\sphinxbfcode{ProgressBar}}
This class is used to display a Progress bar to evaluate the progression
of the string analysis.
\index{\_ProgressBar\_\_cur\_percentage() (lib.progress\_bar.ProgressBar method)}

\begin{fulllineitems}
\phantomsection\label{index:lib.progress_bar.ProgressBar._ProgressBar__cur_percentage}\pysiglinewithargsret{\sphinxbfcode{\_ProgressBar\_\_cur\_percentage}}{}{}
Compute the current percentage of string analysis progression.
\begin{quote}\begin{description}
\item[{Returns}] \leavevmode
Current percentage value.

\item[{Return type}] \leavevmode
int

\end{description}\end{quote}

\end{fulllineitems}

\index{\_ProgressBar\_\_cur\_progression() (lib.progress\_bar.ProgressBar method)}

\begin{fulllineitems}
\phantomsection\label{index:lib.progress_bar.ProgressBar._ProgressBar__cur_progression}\pysiglinewithargsret{\sphinxbfcode{\_ProgressBar\_\_cur\_progression}}{}{}
Build the display of current string analysis progression.
\begin{quote}\begin{description}
\item[{Returns}] \leavevmode
String representing the current progression.

\item[{Return type}] \leavevmode
string

\end{description}\end{quote}

\end{fulllineitems}

\index{\_ProgressBar\_\_display\_bar() (lib.progress\_bar.ProgressBar method)}

\begin{fulllineitems}
\phantomsection\label{index:lib.progress_bar.ProgressBar._ProgressBar__display_bar}\pysiglinewithargsret{\sphinxbfcode{\_ProgressBar\_\_display\_bar}}{}{}
Format and display the progression bar.

\end{fulllineitems}

\index{\_\_init\_\_() (lib.progress\_bar.ProgressBar method)}

\begin{fulllineitems}
\phantomsection\label{index:lib.progress_bar.ProgressBar.__init__}\pysiglinewithargsret{\sphinxbfcode{\_\_init\_\_}}{}{}
Initialize all value needed to display the progress bar.

\end{fulllineitems}

\index{close() (lib.progress\_bar.ProgressBar method)}

\begin{fulllineitems}
\phantomsection\label{index:lib.progress_bar.ProgressBar.close}\pysiglinewithargsret{\sphinxbfcode{close}}{\emph{str}, \emph{endedCorrectly}}{}
This method is called in the end of the progression bar to clean the
display. It also display if the analysis have succeeded or not.

\end{fulllineitems}

\index{init() (lib.progress\_bar.ProgressBar method)}

\begin{fulllineitems}
\phantomsection\label{index:lib.progress_bar.ProgressBar.init}\pysiglinewithargsret{\sphinxbfcode{init}}{\emph{iterable}, \emph{init\_str}}{}
Initialize the progression bar.
\begin{quote}\begin{description}
\item[{Parameters}] \leavevmode\begin{itemize}
\item {} 
\textbf{\texttt{iterable}} (\emph{\texttt{List}}) -- List of strings to analyze.

\item {} 
\textbf{\texttt{init\_str}} (\emph{\texttt{string}}) -- String to display in the begin of the progression
bar.

\end{itemize}

\end{description}\end{quote}

\end{fulllineitems}

\index{update() (lib.progress\_bar.ProgressBar method)}

\begin{fulllineitems}
\phantomsection\label{index:lib.progress_bar.ProgressBar.update}\pysiglinewithargsret{\sphinxbfcode{update}}{}{}
This method increment the progression bar by adding 1 percent.

\end{fulllineitems}


\end{fulllineitems}



\section{parser.py}
\label{index:parser-py}\label{index:module-lib.parser}\index{lib.parser (module)}\index{Parser (class in lib.parser)}

\begin{fulllineitems}
\phantomsection\label{index:lib.parser.Parser}\pysigline{\sphinxstrong{class }\sphinxcode{lib.parser.}\sphinxbfcode{Parser}}
This class is used to perform all tasks concerning the string analysis.
\index{getCmd() (lib.parser.Parser static method)}

\begin{fulllineitems}
\phantomsection\label{index:lib.parser.Parser.getCmd}\pysiglinewithargsret{\sphinxstrong{static }\sphinxbfcode{getCmd}}{\emph{string}}{}
Static method that extract strings detected as a linux command line.
\begin{quote}\begin{description}
\item[{Parameters}] \leavevmode
\textbf{\texttt{string}} (\emph{\texttt{String}}) -- A string to be analyzed.

\item[{Returns}] \leavevmode
None if the string doesn't match or the part of the string
which corresponds.

\item[{Return type}] \leavevmode
string

\end{description}\end{quote}

\end{fulllineitems}

\index{getFormatStr() (lib.parser.Parser static method)}

\begin{fulllineitems}
\phantomsection\label{index:lib.parser.Parser.getFormatStr}\pysiglinewithargsret{\sphinxstrong{static }\sphinxbfcode{getFormatStr}}{\emph{string}}{}
Static method that extract strings detected as a format string.
\begin{quote}\begin{description}
\item[{Parameters}] \leavevmode
\textbf{\texttt{string}} (\emph{\texttt{String}}) -- A string to be analyzed.

\item[{Returns}] \leavevmode
None if the string doesn't match or the part of the string
which corresponds.

\item[{Return type}] \leavevmode
string

\end{description}\end{quote}

\end{fulllineitems}

\index{getId() (lib.parser.Parser static method)}

\begin{fulllineitems}
\phantomsection\label{index:lib.parser.Parser.getId}\pysiglinewithargsret{\sphinxstrong{static }\sphinxbfcode{getId}}{\emph{string}}{}
Static method that extract strings detected as an identifier.
\begin{quote}\begin{description}
\item[{Parameters}] \leavevmode
\textbf{\texttt{string}} (\emph{\texttt{String}}) -- A string to be analyzed.

\item[{Returns}] \leavevmode
None if the string doesn't match or the part of the string
which corresponds.

\item[{Return type}] \leavevmode
string

\end{description}\end{quote}

\end{fulllineitems}

\index{getIp() (lib.parser.Parser static method)}

\begin{fulllineitems}
\phantomsection\label{index:lib.parser.Parser.getIp}\pysiglinewithargsret{\sphinxstrong{static }\sphinxbfcode{getIp}}{\emph{string}}{}
Static method that extract strings detected as an IP address.
\begin{quote}\begin{description}
\item[{Parameters}] \leavevmode
\textbf{\texttt{string}} (\emph{\texttt{String}}) -- A string to be analyzed.

\item[{Returns}] \leavevmode
None if the string doesn't match or the part of the string
which corresponds.

\item[{Return type}] \leavevmode
string

\end{description}\end{quote}

\end{fulllineitems}

\index{getMessage() (lib.parser.Parser static method)}

\begin{fulllineitems}
\phantomsection\label{index:lib.parser.Parser.getMessage}\pysiglinewithargsret{\sphinxstrong{static }\sphinxbfcode{getMessage}}{\emph{string}}{}
Static method that extract strings detected as an english message.
\begin{quote}\begin{description}
\item[{Parameters}] \leavevmode
\textbf{\texttt{string}} (\emph{\texttt{String}}) -- A string to be analyzed.

\item[{Returns}] \leavevmode
None if the string doesn't match or the part of the string
which corresponds.

\item[{Return type}] \leavevmode
string

\end{description}\end{quote}

\end{fulllineitems}

\index{getPath() (lib.parser.Parser static method)}

\begin{fulllineitems}
\phantomsection\label{index:lib.parser.Parser.getPath}\pysiglinewithargsret{\sphinxstrong{static }\sphinxbfcode{getPath}}{\emph{string}}{}
Static method that extract strings detected as a file path.
\begin{quote}\begin{description}
\item[{Parameters}] \leavevmode
\textbf{\texttt{string}} (\emph{\texttt{String}}) -- A string to be analyzed.

\item[{Returns}] \leavevmode
None if the string doesn't match or the part of the string
which corresponds.

\item[{Return type}] \leavevmode
string

\end{description}\end{quote}

\end{fulllineitems}

\index{getSection() (lib.parser.Parser static method)}

\begin{fulllineitems}
\phantomsection\label{index:lib.parser.Parser.getSection}\pysiglinewithargsret{\sphinxstrong{static }\sphinxbfcode{getSection}}{\emph{string}}{}
Static method that extract strings detected as a elf binary section.
\begin{quote}\begin{description}
\item[{Parameters}] \leavevmode
\textbf{\texttt{string}} (\emph{\texttt{String}}) -- A string to be analyzed.

\item[{Returns}] \leavevmode
None if the string doesn't match or the part of the string
which corresponds.

\item[{Return type}] \leavevmode
string

\end{description}\end{quote}

\end{fulllineitems}

\index{getUrl() (lib.parser.Parser static method)}

\begin{fulllineitems}
\phantomsection\label{index:lib.parser.Parser.getUrl}\pysiglinewithargsret{\sphinxstrong{static }\sphinxbfcode{getUrl}}{\emph{string}}{}
Static method that extract strings detected as an URL.
\begin{quote}\begin{description}
\item[{Parameters}] \leavevmode
\textbf{\texttt{string}} (\emph{\texttt{String}}) -- A string to be analyzed.

\item[{Returns}] \leavevmode
None if the string doesn't match or the part of the string
which corresponds.

\item[{Return type}] \leavevmode
string

\end{description}\end{quote}

\end{fulllineitems}

\index{isValidBin() (lib.parser.Parser static method)}

\begin{fulllineitems}
\phantomsection\label{index:lib.parser.Parser.isValidBin}\pysiglinewithargsret{\sphinxstrong{static }\sphinxbfcode{isValidBin}}{\emph{path}}{}
Detect if a file is an elf binary.
\begin{quote}\begin{description}
\item[{Parameters}] \leavevmode
\textbf{\texttt{path}} (\emph{\texttt{string}}) -- Absolute path to an elf binary.

\item[{Returns}] \leavevmode
First value determine if its a binary or not and the second
contains the error message in case of error.

\item[{Return type}] \leavevmode
Boolean, string

\end{description}\end{quote}

\end{fulllineitems}

\index{strings() (lib.parser.Parser static method)}

\begin{fulllineitems}
\phantomsection\label{index:lib.parser.Parser.strings}\pysiglinewithargsret{\sphinxstrong{static }\sphinxbfcode{strings}}{\emph{path}}{}
Extract all readable strings from a binary.
\begin{quote}\begin{description}
\item[{Parameters}] \leavevmode
\textbf{\texttt{path}} -- Path to the binary to analyze.

\item[{Type}] \leavevmode
string

\item[{Returns}] \leavevmode
List that contains all strings.

\item[{Return type}] \leavevmode
List

\end{description}\end{quote}

\end{fulllineitems}


\end{fulllineitems}



\section{database.py}
\label{index:database-py}\label{index:module-lib.database}\index{lib.database (module)}\index{Database (class in lib.database)}

\begin{fulllineitems}
\phantomsection\label{index:lib.database.Database}\pysiglinewithargsret{\sphinxstrong{class }\sphinxcode{lib.database.}\sphinxbfcode{Database}}{\emph{malware\_path}}{}
This class is used to interact with database.
\index{\_\_init\_\_() (lib.database.Database method)}

\begin{fulllineitems}
\phantomsection\label{index:lib.database.Database.__init__}\pysiglinewithargsret{\sphinxbfcode{\_\_init\_\_}}{\emph{malware\_path}}{}
Initialize the database and create a table corresponding to the
malware to analyze.
\begin{quote}\begin{description}
\item[{Parameters}] \leavevmode
\textbf{\texttt{malware\_path}} -- Absolute path to the malware binary.

\end{description}\end{quote}

\end{fulllineitems}

\index{close() (lib.database.Database method)}

\begin{fulllineitems}
\phantomsection\label{index:lib.database.Database.close}\pysiglinewithargsret{\sphinxbfcode{close}}{}{}
Close the database.

\end{fulllineitems}

\index{createEntry() (lib.database.Database method)}

\begin{fulllineitems}
\phantomsection\label{index:lib.database.Database.createEntry}\pysiglinewithargsret{\sphinxbfcode{createEntry}}{\emph{string}, \emph{extract}, \emph{type}}{}
This methods is used by all rules to create an entry in the current
table.
\begin{quote}\begin{description}
\item[{Parameters}] \leavevmode\begin{itemize}
\item {} 
\textbf{\texttt{string}} -- String that has been analyzed.

\item {} 
\textbf{\texttt{extract}} -- Part of the string corresponding to as specific rule.

\item {} 
\textbf{\texttt{type}} -- Type of the rule corresponding to the extracted element.

\end{itemize}

\end{description}\end{quote}

\end{fulllineitems}

\index{getIpAddresses() (lib.database.Database method)}

\begin{fulllineitems}
\phantomsection\label{index:lib.database.Database.getIpAddresses}\pysiglinewithargsret{\sphinxbfcode{getIpAddresses}}{}{}
Return all strings detected as an IP address.
\begin{quote}\begin{description}
\item[{Returns}] \leavevmode
All IP address in the current table.

\item[{Type}] \leavevmode
List

\end{description}\end{quote}

\end{fulllineitems}

\index{getUrls() (lib.database.Database method)}

\begin{fulllineitems}
\phantomsection\label{index:lib.database.Database.getUrls}\pysiglinewithargsret{\sphinxbfcode{getUrls}}{}{}
Return all strings detected as an URL.
\begin{quote}\begin{description}
\item[{Returns}] \leavevmode
All URLs in the current table.

\item[{Return type}] \leavevmode
List

\end{description}\end{quote}

\end{fulllineitems}


\end{fulllineitems}



\section{core.py}
\label{index:module-lib.core}\label{index:core-py}\index{lib.core (module)}\index{Core (class in lib.core)}

\begin{fulllineitems}
\phantomsection\label{index:lib.core.Core}\pysiglinewithargsret{\sphinxstrong{class }\sphinxcode{lib.core.}\sphinxbfcode{Core}}{\emph{malware\_path}, \emph{vt\_key}}{}
class core
\index{\_\_init\_\_() (lib.core.Core method)}

\begin{fulllineitems}
\phantomsection\label{index:lib.core.Core.__init__}\pysiglinewithargsret{\sphinxbfcode{\_\_init\_\_}}{\emph{malware\_path}, \emph{vt\_key}}{}
\end{fulllineitems}

\index{run() (lib.core.Core method)}

\begin{fulllineitems}
\phantomsection\label{index:lib.core.Core.run}\pysiglinewithargsret{\sphinxbfcode{run}}{}{}
Entry point of the MSA functions.
\begin{quote}\begin{description}
\item[{Parameters}] \leavevmode
\textbf{\texttt{self}} (\emph{\texttt{object}}) -- blabla

\item[{Returns}] \leavevmode
If the function success or not.

\item[{Return type}] \leavevmode
Boolean

\end{description}\end{quote}

\end{fulllineitems}


\end{fulllineitems}



\chapter{/lib/allRules/}
\label{index:lib-allrules}

\section{cmd.py}
\label{index:cmd-py}\label{index:module-lib.allRules.cmd}\index{lib.allRules.cmd (module)}\index{Cmd (class in lib.allRules.cmd)}

\begin{fulllineitems}
\phantomsection\label{index:lib.allRules.cmd.Cmd}\pysigline{\sphinxstrong{class }\sphinxcode{lib.allRules.cmd.}\sphinxbfcode{Cmd}}
This class is used to extract linux command line from malware
binaries strings. It also inherit the Rules object used to gather
all shared functions and variables.
\index{\_\_init\_\_() (lib.allRules.cmd.Cmd method)}

\begin{fulllineitems}
\phantomsection\label{index:lib.allRules.cmd.Cmd.__init__}\pysiglinewithargsret{\sphinxbfcode{\_\_init\_\_}}{}{}
Initialize type and info\_msg attributes which respectively represent
the type of extracted information (here command line - cmd), and
the message to display when the rule is initialized.

\end{fulllineitems}

\index{run\_analysis() (lib.allRules.cmd.Cmd method)}

\begin{fulllineitems}
\phantomsection\label{index:lib.allRules.cmd.Cmd.run_analysis}\pysiglinewithargsret{\sphinxbfcode{run\_analysis}}{\emph{string\_list}}{}
This method is used by the core to run analyze and extract strings
matching with the ``command line'' type.
\begin{quote}\begin{description}
\item[{Parameters}] \leavevmode
\textbf{\texttt{string\_list}} (\emph{\texttt{List}}) -- All strings to analyse.

\item[{Returns}] \leavevmode
A list of string without strings previously matched.

\item[{Return type}] \leavevmode
List

\end{description}\end{quote}

\end{fulllineitems}


\end{fulllineitems}



\section{formatStr.py}
\label{index:formatstr-py}\label{index:module-lib.allRules.formatStr}\index{lib.allRules.formatStr (module)}\index{FormatStr (class in lib.allRules.formatStr)}

\begin{fulllineitems}
\phantomsection\label{index:lib.allRules.formatStr.FormatStr}\pysigline{\sphinxstrong{class }\sphinxcode{lib.allRules.formatStr.}\sphinxbfcode{FormatStr}}
This class is used to extract format strings from malware
binaries strings. It also inherit the Rules object used to gather
all shared functions and variables.
\index{\_\_init\_\_() (lib.allRules.formatStr.FormatStr method)}

\begin{fulllineitems}
\phantomsection\label{index:lib.allRules.formatStr.FormatStr.__init__}\pysiglinewithargsret{\sphinxbfcode{\_\_init\_\_}}{}{}
Initialize type and info\_msg attributes which respectively represent
the type of extracted information (here format strings - formatStr),
and the message to display when the rule is initialized.

\end{fulllineitems}

\index{run\_analysis() (lib.allRules.formatStr.FormatStr method)}

\begin{fulllineitems}
\phantomsection\label{index:lib.allRules.formatStr.FormatStr.run_analysis}\pysiglinewithargsret{\sphinxbfcode{run\_analysis}}{\emph{string\_list}}{}
This method is used by the core to run analyze and extract strings
matching with the ``format string'' type.
\begin{quote}\begin{description}
\item[{Parameters}] \leavevmode
\textbf{\texttt{string\_list}} (\emph{\texttt{List}}) -- All strings to analyse.

\item[{Returns}] \leavevmode
A list of string without strings previously matched.

\item[{Return type}] \leavevmode
List

\end{description}\end{quote}

\end{fulllineitems}


\end{fulllineitems}



\section{id.py}
\label{index:id-py}\label{index:module-lib.allRules.id}\index{lib.allRules.id (module)}\index{Id (class in lib.allRules.id)}

\begin{fulllineitems}
\phantomsection\label{index:lib.allRules.id.Id}\pysigline{\sphinxstrong{class }\sphinxcode{lib.allRules.id.}\sphinxbfcode{Id}}
This class is used to extract identifiers (username/password) from
malware binaries strings. It also inherit the Rules object used to
gather all shared functions and variables.
\index{\_\_init\_\_() (lib.allRules.id.Id method)}

\begin{fulllineitems}
\phantomsection\label{index:lib.allRules.id.Id.__init__}\pysiglinewithargsret{\sphinxbfcode{\_\_init\_\_}}{}{}
Initialize type and info\_msg attributes which respectively represent
the type of extracted information (here identifiers - Id),
and the message to display when the rule is initialized.

\end{fulllineitems}

\index{run\_analysis() (lib.allRules.id.Id method)}

\begin{fulllineitems}
\phantomsection\label{index:lib.allRules.id.Id.run_analysis}\pysiglinewithargsret{\sphinxbfcode{run\_analysis}}{\emph{string\_list}}{}
This method is used by the core to run analyze and extract strings
matching with the ``identifiers'' type.
\begin{quote}\begin{description}
\item[{Parameters}] \leavevmode
\textbf{\texttt{string\_list}} (\emph{\texttt{List}}) -- All strings to analyse.

\item[{Returns}] \leavevmode
A list of string without strings previously matched.

\item[{Return type}] \leavevmode
List

\end{description}\end{quote}

\end{fulllineitems}


\end{fulllineitems}



\section{ipAddr.py}
\label{index:ipaddr-py}\label{index:module-lib.allRules.ipAddr}\index{lib.allRules.ipAddr (module)}\index{IpAddr (class in lib.allRules.ipAddr)}

\begin{fulllineitems}
\phantomsection\label{index:lib.allRules.ipAddr.IpAddr}\pysigline{\sphinxstrong{class }\sphinxcode{lib.allRules.ipAddr.}\sphinxbfcode{IpAddr}}
This class is used to extract IP addresses from malware
binaries strings. It also inherit the Rules object used to gather
all shared functions and variables.
\index{\_\_init\_\_() (lib.allRules.ipAddr.IpAddr method)}

\begin{fulllineitems}
\phantomsection\label{index:lib.allRules.ipAddr.IpAddr.__init__}\pysiglinewithargsret{\sphinxbfcode{\_\_init\_\_}}{}{}
Initialize type and info\_msg attributes which respectively represent
the type of extracted information (here format ip addresses - IpAddr),
and the message to display when the rule is initialized.

\end{fulllineitems}

\index{run\_analysis() (lib.allRules.ipAddr.IpAddr method)}

\begin{fulllineitems}
\phantomsection\label{index:lib.allRules.ipAddr.IpAddr.run_analysis}\pysiglinewithargsret{\sphinxbfcode{run\_analysis}}{\emph{string\_list}}{}
This method is used by the core to run analyze and extract strings
matching with the ``ip addresses'' type.
\begin{quote}\begin{description}
\item[{Parameters}] \leavevmode
\textbf{\texttt{string\_list}} (\emph{\texttt{List}}) -- All strings to analyse.

\item[{Returns}] \leavevmode
A list of string without strings previously matched.

\item[{Return type}] \leavevmode
List

\end{description}\end{quote}

\end{fulllineitems}


\end{fulllineitems}



\section{message.py}
\label{index:message-py}\label{index:module-lib.allRules.message}\index{lib.allRules.message (module)}\index{Msg (class in lib.allRules.message)}

\begin{fulllineitems}
\phantomsection\label{index:lib.allRules.message.Msg}\pysigline{\sphinxstrong{class }\sphinxcode{lib.allRules.message.}\sphinxbfcode{Msg}}
This class is used to extract english message from malware
binaries strings. It also inherit the Rules object used to gather
all shared functions and variables.
\index{\_\_init\_\_() (lib.allRules.message.Msg method)}

\begin{fulllineitems}
\phantomsection\label{index:lib.allRules.message.Msg.__init__}\pysiglinewithargsret{\sphinxbfcode{\_\_init\_\_}}{}{}
Initialize type and info\_msg attributes which respectively represent
the type of extracted information (here format messages - Msg),
and the message to display when the rule is initialized.

\end{fulllineitems}

\index{run\_analysis() (lib.allRules.message.Msg method)}

\begin{fulllineitems}
\phantomsection\label{index:lib.allRules.message.Msg.run_analysis}\pysiglinewithargsret{\sphinxbfcode{run\_analysis}}{\emph{string\_list}}{}
This method is used by the core to run analyze and extract strings
matching with the ``message'' type.
\begin{quote}\begin{description}
\item[{Parameters}] \leavevmode
\textbf{\texttt{string\_list}} (\emph{\texttt{List}}) -- All strings to analyse.

\item[{Returns}] \leavevmode
A list of string without strings previously matched.

\item[{Return type}] \leavevmode
List

\end{description}\end{quote}

\end{fulllineitems}


\end{fulllineitems}



\section{path.py}
\label{index:module-lib.allRules.path}\label{index:path-py}\index{lib.allRules.path (module)}\index{Path (class in lib.allRules.path)}

\begin{fulllineitems}
\phantomsection\label{index:lib.allRules.path.Path}\pysigline{\sphinxstrong{class }\sphinxcode{lib.allRules.path.}\sphinxbfcode{Path}}
This class is used to extract file path from malware
binaries strings. It also inherit the Rules object used to gather
all shared functions and variables.
\index{\_\_init\_\_() (lib.allRules.path.Path method)}

\begin{fulllineitems}
\phantomsection\label{index:lib.allRules.path.Path.__init__}\pysiglinewithargsret{\sphinxbfcode{\_\_init\_\_}}{}{}
Initialize type and info\_msg attributes which respectively represent
the type of extracted information (here file path - path),
and the message to display when the rule is initialized.

\end{fulllineitems}

\index{run\_analysis() (lib.allRules.path.Path method)}

\begin{fulllineitems}
\phantomsection\label{index:lib.allRules.path.Path.run_analysis}\pysiglinewithargsret{\sphinxbfcode{run\_analysis}}{\emph{string\_list}}{}
This method is used by the core to run analyze and extract strings
matching with the ``file path'' type.
\begin{quote}\begin{description}
\item[{Parameters}] \leavevmode
\textbf{\texttt{string\_list}} (\emph{\texttt{List}}) -- All strings to analyse.

\item[{Returns}] \leavevmode
A list of string without strings previously matched.

\item[{Return type}] \leavevmode
List

\end{description}\end{quote}

\end{fulllineitems}


\end{fulllineitems}



\section{section.py}
\label{index:section-py}\label{index:module-lib.allRules.section}\index{lib.allRules.section (module)}\index{Section (class in lib.allRules.section)}

\begin{fulllineitems}
\phantomsection\label{index:lib.allRules.section.Section}\pysigline{\sphinxstrong{class }\sphinxcode{lib.allRules.section.}\sphinxbfcode{Section}}
This class is used to extract binaries elf sections from malware
binaries strings. It also inherit the Rules object used to gather
all shared functions and variables.
\index{\_\_init\_\_() (lib.allRules.section.Section method)}

\begin{fulllineitems}
\phantomsection\label{index:lib.allRules.section.Section.__init__}\pysiglinewithargsret{\sphinxbfcode{\_\_init\_\_}}{}{}
Initialize type and info\_msg attributes which respectively represent
the type of extracted information (here elf sections - section),
and the message to display when the rule is initialized.

\end{fulllineitems}

\index{run\_analysis() (lib.allRules.section.Section method)}

\begin{fulllineitems}
\phantomsection\label{index:lib.allRules.section.Section.run_analysis}\pysiglinewithargsret{\sphinxbfcode{run\_analysis}}{\emph{string\_list}}{}
This method is used by the core to run analyze and extract strings
matching with the ``elf section'' type.
\begin{quote}\begin{description}
\item[{Parameters}] \leavevmode
\textbf{\texttt{string\_list}} (\emph{\texttt{List}}) -- All strings to analyse.

\item[{Returns}] \leavevmode
A list of string without strings previously matched.

\item[{Return type}] \leavevmode
List

\end{description}\end{quote}

\end{fulllineitems}


\end{fulllineitems}



\section{symbol.py}
\label{index:symbol-py}\label{index:module-lib.allRules.symbol}\index{lib.allRules.symbol (module)}\index{Symbol (class in lib.allRules.symbol)}

\begin{fulllineitems}
\phantomsection\label{index:lib.allRules.symbol.Symbol}\pysiglinewithargsret{\sphinxstrong{class }\sphinxcode{lib.allRules.symbol.}\sphinxbfcode{Symbol}}{\emph{malware\_path}}{}
This class is used to extract binaries elf symbols from malware
binaries strings. It also inherit the Rules object used to gather
all shared functions and variables.
\index{\_\_init\_\_() (lib.allRules.symbol.Symbol method)}

\begin{fulllineitems}
\phantomsection\label{index:lib.allRules.symbol.Symbol.__init__}\pysiglinewithargsret{\sphinxbfcode{\_\_init\_\_}}{\emph{malware\_path}}{}
Initialize type and info\_msg attributes which respectively represent
the type of extracted information (here elf symbols - symbol),
and the message to display when the rule is initialized.
This class also contains a malware\_path value containing the absolute
path of the analyzed binary.
\begin{quote}\begin{description}
\item[{Parameters}] \leavevmode
\textbf{\texttt{malware\_path}} -- Absolute path of the analyzed binary.

\item[{Type}] \leavevmode
string

\end{description}\end{quote}

\end{fulllineitems}

\index{run\_analysis() (lib.allRules.symbol.Symbol method)}

\begin{fulllineitems}
\phantomsection\label{index:lib.allRules.symbol.Symbol.run_analysis}\pysiglinewithargsret{\sphinxbfcode{run\_analysis}}{\emph{string\_list}}{}
This method is used by the core to run analyze and extract strings
matching with the ``elf symbols'' type.
\begin{quote}\begin{description}
\item[{Parameters}] \leavevmode
\textbf{\texttt{string\_list}} (\emph{\texttt{List}}) -- All strings to analyse.

\item[{Returns}] \leavevmode
A list of string without strings previously matched.

\item[{Return type}] \leavevmode
List

\end{description}\end{quote}

\end{fulllineitems}


\end{fulllineitems}



\section{undefined.py}
\label{index:undefined-py}\label{index:module-lib.allRules.undefined}\index{lib.allRules.undefined (module)}\index{Undefined (class in lib.allRules.undefined)}

\begin{fulllineitems}
\phantomsection\label{index:lib.allRules.undefined.Undefined}\pysigline{\sphinxstrong{class }\sphinxcode{lib.allRules.undefined.}\sphinxbfcode{Undefined}}
This class is used to extract all undefined strings from malware
binaries strings. It also inherit the Rules object used to gather
all shared functions and variables.
\index{\_\_init\_\_() (lib.allRules.undefined.Undefined method)}

\begin{fulllineitems}
\phantomsection\label{index:lib.allRules.undefined.Undefined.__init__}\pysiglinewithargsret{\sphinxbfcode{\_\_init\_\_}}{}{}
Initialize type and info\_msg attributes which respectively represent
the type of extracted information (here undefined strings - undefined),
and the message to display when the rule is initialized.

\end{fulllineitems}

\index{run\_analysis() (lib.allRules.undefined.Undefined method)}

\begin{fulllineitems}
\phantomsection\label{index:lib.allRules.undefined.Undefined.run_analysis}\pysiglinewithargsret{\sphinxbfcode{run\_analysis}}{\emph{string\_list}}{}
This method is used by the core to run analyze and extract strings
matching with the ``undefined'' type.
\begin{quote}\begin{description}
\item[{Parameters}] \leavevmode
\textbf{\texttt{string\_list}} (\emph{\texttt{List}}) -- All strings to analyse.

\item[{Returns}] \leavevmode
A list of string without strings previously matched.

\item[{Return type}] \leavevmode
List

\end{description}\end{quote}

\end{fulllineitems}


\end{fulllineitems}



\section{url.py}
\label{index:module-lib.allRules.url}\label{index:url-py}\index{lib.allRules.url (module)}\index{Url (class in lib.allRules.url)}

\begin{fulllineitems}
\phantomsection\label{index:lib.allRules.url.Url}\pysigline{\sphinxstrong{class }\sphinxcode{lib.allRules.url.}\sphinxbfcode{Url}}
This class is used to extract URLs from malware
binaries strings. It also inherit the Rules object used to gather
all shared functions and variables.
\index{\_\_init\_\_() (lib.allRules.url.Url method)}

\begin{fulllineitems}
\phantomsection\label{index:lib.allRules.url.Url.__init__}\pysiglinewithargsret{\sphinxbfcode{\_\_init\_\_}}{}{}
Initialize type and info\_msg attributes which respectively represent
the type of extracted information (here URLs - url),
and the message to display when the rule is initialized.

\end{fulllineitems}

\index{run\_analysis() (lib.allRules.url.Url method)}

\begin{fulllineitems}
\phantomsection\label{index:lib.allRules.url.Url.run_analysis}\pysiglinewithargsret{\sphinxbfcode{run\_analysis}}{\emph{string\_list}}{}
This method is used by the core to run analyze and extract strings
matching with the ``url'' type.
\begin{quote}\begin{description}
\item[{Parameters}] \leavevmode
\textbf{\texttt{string\_list}} (\emph{\texttt{List}}) -- All strings to analyse.

\item[{Returns}] \leavevmode
A list of string without strings previously matched.

\item[{Return type}] \leavevmode
List

\end{description}\end{quote}

\end{fulllineitems}


\end{fulllineitems}



\renewcommand{\indexname}{Python Module Index}
\begin{theindex}
\def\bigletter#1{{\Large\sffamily#1}\nopagebreak\vspace{1mm}}
\bigletter{l}
\item {\texttt{lib.allRules.cmd}}, \pageref{index:module-lib.allRules.cmd}
\item {\texttt{lib.allRules.formatStr}}, \pageref{index:module-lib.allRules.formatStr}
\item {\texttt{lib.allRules.id}}, \pageref{index:module-lib.allRules.id}
\item {\texttt{lib.allRules.ipAddr}}, \pageref{index:module-lib.allRules.ipAddr}
\item {\texttt{lib.allRules.message}}, \pageref{index:module-lib.allRules.message}
\item {\texttt{lib.allRules.path}}, \pageref{index:module-lib.allRules.path}
\item {\texttt{lib.allRules.section}}, \pageref{index:module-lib.allRules.section}
\item {\texttt{lib.allRules.symbol}}, \pageref{index:module-lib.allRules.symbol}
\item {\texttt{lib.allRules.undefined}}, \pageref{index:module-lib.allRules.undefined}
\item {\texttt{lib.allRules.url}}, \pageref{index:module-lib.allRules.url}
\item {\texttt{lib.core}}, \pageref{index:module-lib.core}
\item {\texttt{lib.database}}, \pageref{index:module-lib.database}
\item {\texttt{lib.parser}}, \pageref{index:module-lib.parser}
\item {\texttt{lib.progress\_bar}}, \pageref{index:module-lib.progress_bar}
\item {\texttt{lib.rules}}, \pageref{index:module-lib.rules}
\item {\texttt{lib.vt}}, \pageref{index:module-lib.vt}
\indexspace
\bigletter{m}
\item {\texttt{msa}}, \pageref{index:module-msa}
\end{theindex}

\renewcommand{\indexname}{Index}
\printindex
\end{document}
