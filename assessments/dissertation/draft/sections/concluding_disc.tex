\paragraph{}

In this paper, we presented the usage and the installation of an SSH honeypot. As we have 
seen this tool allow to easily collect basics attacks on Linux server quickly without
using a lot of resources. Indeed, only the cost is the server which is currently not really
expensive especially if we choose a Virtual machine as a server.
However, it is important to notice that honeypot is a known technique by attackers and
given the fact that honeypots are software it is impossible to be sure that they are safe
to deploy on a network or on a server containing sensitive data. That is why I choose the
Cowrie honeypot which is a maintained project with an available code source which reduces 
the risk of vulnerability, even it is impossible to be sure that there is no security hole
on it.

By using this honeypot I have been able to analyse and understand attackers behaviours that
attack Linux Server through the SSH port. I was mainly surprise of the simplicity of their
attack scenario. Indeed, some basic security standard can avoid the totality of intrusions
that only apply a basic brute force attack, with login attempts by using most commonly used
usernames and passwords. First of all, it is possible to change default usernames and choose
strong passwords with alphanumerics characters in upper and lower case associate to symbols.
Moreover, it is possible to install services on the server such as fail2ban, which consists
of blocking after a configured number of fail login attempts. Finally, it could also be a good
practice to change the SSH port, initially port 22 to another.

I have also been able to collect and analyse a sample of malware currently used by attackers.
Based on their number I could not be able to perform a specific analysis for each of them.
Consequently, I started to use Cuckoo malware analysis framework, however, this tool was
mainly focused on Windows malware analysis. Despite the usage a GitHub project that 
improvement Cuckoo in order to analyse Linux malware, I was not able to configure it correctly.
Hence, I used Limon another malware analysis tool but focused on Linux malware. I applied
it on the first collected malware, but the quantity of generated data was to large, due
to its static and dynamic analysis. However, in its report, I found interesting to analyse
extracted strings.
Based on those strings I developed the malware strings analyser tool with the goal to be 
re-used in order to find patterns on those strings that can be crossed to a data mining 
algorithm. This could improve the malware pre-analysis techniques currently used.

Finally, in the way as the HoneyNets project, all of the collected data are aimed to update
known attackers techniques and improve knowledge about Linux server security in order
to generate adapted security protection against those specific attacks.
Indeed, this project contributes to the infinite race between attackers and security researchers,
in which attackers are still one step ahead of us.
