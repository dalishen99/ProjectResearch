\subsection{Attacks overview}

\paragraph{} % 419

The Cowrie honeypot has been put in place the 16th March 2016 and will continue to run
until the end of the dissertation, the 12th September 2016. During this period a number
of 82388 SSH login attempts have been recorded. Those unauthorised connections were detected
as being executed from 2706 distinct source IP addresses. However, some of those IP addresses
have been more active than others. Indeed, 20\% of all of those connections have been performed
by 10 distinct IP addresses. Moreover, attacks from these 10 IP addresses were from a different
part of the world such as China, United States, France, Japan and Republic of Korea.
Hence, this list of country is finally the top five of connections per country during this
period of attacks on the honeypot.

Furthermore, the cowrie honeypot has been also able to determine the version and type of the
SSH connection library and tool use to initiate and SSH connection on our server.
Hence, we can see that attackers use mostly the SSH-2.0 libssh in multiple version such as
the 1.6.0 or 1.4.3. These libraries can be used in multiple operating systems such as
Linux and Windows. In the same way, the honeypot has detected that in the top 10 of ssh clients
the Putty software has been used. This tool originally developed for Windows is now available
on Linux systems. Consequently, this information cannot be reused to determine the operating
system of the attackers I had hoped.

Kippo graph has also collected all shell inputs written by attackers once they were connected.
Concerning this post-compromise part, a total of 60023 commands has been inputted with
only 1011 distinct commands. Those data are really interesting because the top 10 of most
commonly used commands represent 49\% of all detected inputs. The first command in this
ranking is "mkdir /tmp/.xs/" which consists of creating a hidden directory into the temporary directory of the targeted Linux system. This command has been used a total of 10875 times,
consequently, we can easily suppose that this has been executed by a bot. This time of script
consists of automating the execution of given command lines. In the same way, we can see that
the top 5 of failed command, which are all invalid shell command lines, represent a total
of 10834 commands. Consequently, it is difficult to believe that a human attacker has been
able to make this amount of mistakes. It is more likely that this has been performed by a
script containing an error.

\subsection{Authentication attempts}

\paragraph{} % 324

On the 82388 attackers SSH login attempts recorded, 66\% of them were considered as 
successful connection because a full shell access was granted to them. This step of 
connection attempts on our SSH server can be considered as a brute force attack. Indeed,
attackers through their scripts tried successively a list of common usernames and passwords
in order to access the server remotely.

This part has been really interesting to analyse because the cowrie honeypot as collected
and stored all of those identifiers in its database. Those data have allowed the malware
strings analyser to extract known identifiers from hard coded strings on malware binaries.
Moreover, these data confirm relevance to use complex passwords other than simple common
strings such as "admin", "root" or "12345". Indeed these last 3 passwords are the top 3 of 
tested passwords on the SSH server. In the same way, usernames are also important to choose,
indeed we can see that attackers try default usernames (and also default passwords) from
known networks devices such as "git", "oracle" and "postgres", which correspond to known
tools and databases.

Finally, we can observe, that the brute force attack here consists to use a short list of known 
identifiers and not try to modify character by character in order to find the correct username/
password. Indeed, attackers only try a fixed number of identifiers by IP addresses before achieving
to access to our SSH server. Consequently, we can easily suppose that attackers only try
to access to a server with weak authentication identifiers standards and change to another if 
usernames and passwords are not in their dictionaries. This allows attackers to browse all
public addresses one by one and testing in case of an SSH port open all of their passwords.
This attack is really simple, however, the probability to obtain an access to a server by this
the technique is not low because of the large range of public IP addresses available.

\subsection{Malicious downloads}

\paragraph{} % 299

Once attackers were logged on the server through SSH, a common task was to download what we
can call a malware deployment script. Indeed, attackers usually download a shell script by
using mostly the "wget" command line, which is used to get a content from a given URL, in our
case a script. The advantage of this script is that it could be run on every Linux server 
without downloading or installing additionals packages.
Then, the attackers run this shell script that will download binaries with the wget command,
grant execution rules to the binary by using the "chmod" command line and run it. This sequence
of command is re-executed for each compiled version of the malware.

This way to compromised a Linux is surprisingly really simple. Indeed, once the script is 
downloaded it will download and run each malware, compiled for most common processor architecture
in order to be able to correctly run. However, the script does not try to get more information
of the system in order to download and run the appropriate malware.

All of these downloaded versions of malware were those that I chose to analyse with the 
malware strings analyser script. I decided to re-download them manually from each 
malware deployment scripts downloaded by attackers on my server. I had to modify these
script shell manually to only keep the downloading part. However, I was not able to 
directly perform this task just after the attack because of the random hour when those events
occurred. Consequently, when I try to re-download all binaries from attackers scripts some URLs
pointing to binary to download had expired.
This shows us that attackers have multiple servers used to host their malware and deployment
scripts, but their usage expired probably in order to avoid to be easily tracked.

\subsection{Malware binaries analysis}

\paragraph{} % 437

During the entire period of the project research, I have collected a large sample of malware
binary. All of them were known malware and were usually the same. Indeed, the majority of them
were detected as the malware called "GayFgt". This has been possible to be detected by using
the Virus Total option of the malware strings analyser script.

Furthermore, the larger part of the analysis was not focused on a full static or dynamic 
analysis of the malware as usual. Indeed, I decided to investigate on the content of strings
hard coded in those malware binaries.
First of all, it is important to recall that malware can be packed, which consists of encrypting a the malware. Then, once it is executed the malware will decrypt its code itself dynamically
and run normally. The current version of the malware strings analyser can detect that it has
been packed but it will abort the execution and it does not try to unpack it with known
unpacking techniques (for instance UPX packing method). In the same way, some collected binaries
were stripped which results to loose strings containing symbols information about the malware.
As for the packing issue, the first version of the malware string analyser will abort its analysis
in the case of strip malware.

Concerning the analysis of not stripped and packed malware, the malware strings analyser has
been able to extract some useful information about each malware. I have been able
to collect some other URLs and IP addresses stored in this malware.
Indeed, found URLs were used to download the new version of the malware from a remote server.
This has been confirmed by using the Virus Total option of the malware strings analyser which
analysed each URLs found and confirmed that those URLs were known as storing malicious content.
Furthermore, IP addresses found have been detected has remote command and control server by
using the same Virus Total option of the malware strings analyser tool.

Finally, other interesting categories were the potential displayed message and format strings.
Indeed, each of them contains strings that explicitly talk about TCP and UDP Flooding, followed
by a format string making think of an IP address. This and other strings resembling a list
of remote commands and error message, suggests that these malware were connected to a 
remote command and control server in order to perform DDoS attacks with TCP and UPD packets
against a given targeted IP address. Some specific research concerning the "GayFgt" malware
have confirmed that the DDoS attacks was one of its functionality and confirmed my assumption
by using only a static string analysis.
