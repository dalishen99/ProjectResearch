\textsc{\LARGE Abstract}\\[0.5cm]

\paragraph{}

Attacks against servers and in particular Linux servers are becoming increasingly
common. However, their protection is necessary in order to maintain the normal functioning
of all services that all users who have a device connected use. Indeed, websites, mobiles app
emails, and a lot of other services use servers in order to work.
However, it is important to notice that attackers  are usually one step ahead of security
researchers. But some techniques are used to understand attackers behaviours in order to 
develop a new way to protect systems against them. A honeypot is one of those techniques that
emulate vulnerable servers designed to collect attacks and analysed them in real time.
Consequently, this paper will describe the normal process of using a Cowrie SSH honeypot,
that has been used during 6 months in order to collect attacks information and samples of
malicious files. Moreover, this paper will also present a pre-analysis static technique
to analyse malware, only based on their strings, by using gathered malware sample. 


\newpage