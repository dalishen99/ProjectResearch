\textsc{\LARGE Abstract}\\[0.5cm]

\paragraph{}

Attacks against servers and in particular Linux servers are becoming increasingly
common. However, their protection is necessary in order to maintain the normal functioning
of all services that all users, who have a device connected, use. Indeed, websites, mobile
apps emails, and a lot of other services need servers in order to work.
Moreover, it is important to notice that attackers  are usually one step ahead of security
researchers. But some techniques are used to understand attackers behaviours in order to 
develop new ways to protect systems against them. A honeypot is one of those techniques.
It consists of emulating vulnerable servers designed to collect attacks and analysed them
in real time.
Consequently, this paper will describe the usage of a Cowrie SSH honeypot,
that has been put in place and run during 6 months in order to collect attacks information
and samples of malicious files. Moreover, this paper will also present a pre-analysis
static technique to examine malware. This analysis method has been performed by using
hard-coded strings from gathered malware sample. 

\newpage